\section{Introduccion}

\textit{Dc Construcciones}, una empresa dedicada al apoyo de proyectos de construccion, se ve frente al desborde propio de su constante demanda y crecimiento. Mediante las herramientas vistas en el curso se nos pide modelar y especificar una posible solucion para este escenario. Para ello hemos realizado el relevamiento correspondiente durante una charla inicial con los gerentes de dicha empresa, gente entusiasta y emprendedora cuyo interes hace foco en una solida mantenibilidad y escalabilidad. Teniendo esto en cuenta hemos desarrollado la presente propuesta.

\subsection{Contexto}

Con el objeto de satisfacer las exigencias surgidas del relevamiento, consideramos relevantes los siguientes puntos que deberan ser tenidos en cuenta de aqui en mas:

\begin{itemize}

	\item Los clientes deben poder evaluar y manifestar su conformidad sobre los avances de cada proyecto.
	
	\item Los gerentes deben estar al tanto sobre la conformidad del cliente y el progreso de cada proyecto.
	
  \item Puede darse el caso de que haya clientes que no dispongan de conexion a internet, estos no deben quedar fuera del alcance de la empresa.
  
	\item A cada proyecto se le debe asignar un PM, que dada cierta disconformidad por parte del cliente podria llegar a ser desplazado.
	
	\item La lista de PMs solo puede ser modificada por los gerentes.
	
	\item Una vez asignado un PM a cierto proyecto, este sera el resposable de seleccionar un proveedor de entre la lista de proveedores de la empresa.
	
	\item Siempre debe asegurarse de que el proveedor a contratar tenga el seguro de caucion al dia.
	
  \item Siempre se debe poder actualizar la \textit{base de dato} de los proveedores; estas son el bien mas importante de la empresa y por ende es escencial permitir su expansion y mantenimiento. 

  \item Los PMs son los que se encargan de acordar el alcance y condiciones con el cliente, para luego enviar una propuesta a los gerentes y ser evaluada.
	
  \item Los PMs deben reportar avances y novedades al sistema acerca de sus proyectos asignados.

  \item El cliente puede llegar a pedir adicionales en medio de la obra.

  \item Para empezar el proyecto tanto el cliente como el proveedor deben firmar un contrato. 

\end{itemize}
